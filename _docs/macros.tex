\usepackage[margin = 1in]{geometry}

\usepackage[T1]{fontenc}
\usepackage{listings}
\usepackage{xcolor}
\usepackage{tikz}
\usepackage{framed}
\usepackage[most]{tcolorbox}
\usepackage{multicol}
\usepackage{graphicx}
\usepackage{float}
\usepgflibrary{patterns} % LaTeX and plain TeX and pure pgf
\usepgflibrary[patterns] % ConTeXt and pure pgf
\usetikzlibrary{patterns} % LaTeX and plain TeX when using TikZ
\usetikzlibrary[patterns] % ConTeXt when using TikZ

\newcommand{\ctbox}[1]{
    \begin{center}
  \tcbox[colback=gray!10, boxrule=.3mm]{#1}
\end{center}
}

\newcommand{\freestyletape}[8]{
  \draw [fill=tapeBg, tapeBg] (#1, #2) -- (#3, #4) [in=180, out=0] to (#7, #8) -- (#5, #6)  [in=0, out=180] to cycle;

   \draw[tapeBorder, line width=0.5pt, in=180, out=0] (#1, #2) to (#5, #6);
  \draw[tapeBorder, line width=0.5pt, in=180, out=0] (#3, #4) to (#7, #8);
}

\newcommand{\id}[3]{
  \node [nodestyle] (ida#1) at (#2,#3) {};
  \node [nodestyle] (idb#1) at (#2 + 1,#3) {};
  \draw (ida#1) -- (idb#1);
}

% fresh posx posy scaley
\newcommand{\swap}[4]{
 \node [nodestyle] (swapa1) at (#2,#3) {};
 \node [nodestyle] (swapb1) at (#2,#3 + #4) {};
 \node [nodestyle] (swapc1) at (#2+1,#3) {};
 \node [nodestyle] (swapd1) at (#2+1,#3 + #4) {};

 \draw [in=180, out=0] (swapa1) to (swapd1);
 \draw [in=-180, out=0] (swapb1) to (swapc1);
}

% posx posy width height
\newcommand{\tape}[4]{
  \draw [fill=tapeBg, tapeBg] (#1, #2) -- (#1+#3, #2) -- (#1+#3, #2+#4) -- (#1, #2+#4) -- cycle;

  \draw[tapeBorder, line width=0.5pt] (#1, #2) -- (#1+#3, #2);
  \draw[tapeBorder, line width=0.5pt] (#1+#3, #2+#4) -- (#1, #2+#4);
}

% fresh posx posy arity-1 coarity-1 name otimesdist
\newcommand{\gen}[7]{
  \pgfmathsetmacro\arminone{#4};
  \pgfmathsetmacro\coarminone{#5};
  \pgfmathsetmacro\otimesdist{#7};

  \pgfmathsetmacro\arity{#4 + 1};
  \pgfmathsetmacro\coarity{#5 + 1};

\pgfmathparse{%
  (\arity>\coarity)
    ? \arminone * \otimesdist
    : \coarminone * \otimesdist
}%
\let\height\pgfmathresult


  \node [boxstyle] (a) at (#2 + 1,#3 + \height / 2) {#6};

  \pgfmathparse{%
  (\arity - 1) / 2 * \otimesdist
}%
\let\arshift\pgfmathresult

\pgfmathparse{%
  (\coarity - 1) / 2 * \otimesdist
  }%
\let\coarshift\pgfmathresult

   \foreach \i in {0,...,\arminone}
  {
    \node [nodestyle] (#1in\i) at (#2, #3 + \i * \otimesdist + \height / 2 - \arshift) {};

    % Correct conditional computation
     % Use TeX conditional to avoid division by zero
    \ifnum\arminone=0
      \def\angle{180}
    \else
      \pgfmathsetmacro{\angle}{(180 / \arminone) * -\i - 90}
    \fi

    \draw[in=0, out=\angle] (a) to (#1in\i);
  }

  \foreach \i in {0,...,\coarminone}
  {
    \node [nodestyle] (#1out\i) at (#2 + 2, #3 + \i * \otimesdist + \height / 2 - \coarshift) {};

    % Correct conditional computation
     % Use TeX conditional to avoid division by zero
    \ifnum\coarminone=0
      \def\angle{0}
    \else
      \pgfmathsetmacro{\angle}{(180 / \coarminone) * \i - 90}
    \fi

    \draw[in=180, out=\angle] (a) to (#1out\i);
  }
}

% posx posy n1 n2 oplusdist otimesdist tapepadding width
\newcommand{\swaptape}[8]{
  \pgfmathsetmacro{\posx}{#1}
  \pgfmathsetmacro{\posy}{#2}
  \pgfmathsetmacro{\none}{#3}
  \pgfmathsetmacro{\ntwo}{#4}
  \pgfmathsetmacro{\oplusdist}{#5}
  \pgfmathsetmacro{\otimesdist}{#6}
  \pgfmathsetmacro{\tapepadding}{#7}
  \pgfmathsetmacro{\width}{#8}

   \ifnum\none=0
      \pgfmathsetmacro{\sizeone}{2 * \tapepadding}
    \else
      \pgfmathsetmacro{\sizeone}{(\none - 1) * \otimesdist + (2 * \tapepadding)}
    \fi

    \ifnum\ntwo=0
      \pgfmathsetmacro{\sizetwo}{2 * \tapepadding}
    \else
      \pgfmathsetmacro{\sizetwo}{(\ntwo - 1) * \otimesdist + (2 * \tapepadding)}
    \fi

    \pgfmathsetmacro{\twobasebotx}{\posx}
    \pgfmathsetmacro{\twobaseboty}{\posy}

    \pgfmathsetmacro{\twoceilbotx}{\posx}
    \pgfmathsetmacro{\twoceilboty}{\posy + \sizetwo}

    \pgfmathsetmacro{\twobasetopx}{\posx + \width}
    \pgfmathsetmacro{\twobasetopy}{\posy + \sizeone + \oplusdist}

    \pgfmathsetmacro{\twoceiltopx}{\posx + \width}
    \pgfmathsetmacro{\twoceiltopy}{\posy + \sizeone + \oplusdist + \sizetwo}

    \pgfmathsetmacro{\onebasebotx}{\posx + \width}
    \pgfmathsetmacro{\onebaseboty}{\posy}

    \pgfmathsetmacro{\oneceilbotx}{\posx + \width}
    \pgfmathsetmacro{\oneceilboty}{\posy + \sizeone}

    \pgfmathsetmacro{\onebasetopx}{\posx}
    \pgfmathsetmacro{\onebasetopy}{\posy + \sizetwo + \oplusdist}

    \pgfmathsetmacro{\oneceiltopx}{\posx}
    \pgfmathsetmacro{\oneceiltopy}{\posy + \sizetwo + \oplusdist + \sizeone}

    \draw [fill=tapeBg, tapeBg, in=180, out=0] (\twobasebotx, \twobaseboty) to (\twobasetopx, \twobasetopy) --  (\twoceiltopx, \twoceiltopy) [in=0, out=180] to (\twoceilbotx, \twoceilboty) -- cycle;

    \draw [fill=tapeBg, tapeBg, in=0, out=180] (\onebasebotx, \onebaseboty) to (\onebasetopx, \onebasetopy) --  (\oneceiltopx, \oneceiltopy) [in=180, out=0] to (\oneceilbotx, \oneceilboty) -- cycle;

    \draw[tapeBorder, in=180, out=0] (\twobasebotx, \twobaseboty) to (\twobasetopx, \twobasetopy);
    \draw[tapeBorder, in=180, out=0] (\twoceilbotx, \twoceilboty) to (\twoceiltopx, \twoceiltopy);


    \draw[tapeBorder, in=0, out=180] (\onebasebotx, \onebaseboty) to (\onebasetopx, \onebasetopy);
    \draw[tapeBorder, in=0, out=180] (\oneceilbotx, \oneceilboty) to (\oneceiltopx, \oneceiltopy);


    \foreach \i in {0,...,\ntwo}
    {
      \pgfmathsetmacro{\iminone}{\i - 1}
      \ifnum\i=0
    \else
      \draw[in=180, out=0] (\twobasebotx, \twobaseboty + \iminone * \otimesdist + \tapepadding) to (\twobasetopx, \twobasetopy + \iminone * \otimesdist + \tapepadding);
    \fi
    }

    \foreach \i in {0,...,\none}
    {
    \pgfmathsetmacro{\iminone}{\i - 1}
     \ifnum\i=0
    \else
      \draw[in=0, out=180] (\oneceilbotx, \onebaseboty + \iminone * \otimesdist + \tapepadding) to   (\onebasetopx, \onebasetopy + (\iminone * \otimesdist + \tapepadding);
    \fi



    }


}


\colorlet{tapeBg}{red!30}
\colorlet{tapeBorder}{red!60}

\tikzset{
nodestyle/.style={circle, minimum size=0pt, inner sep=0pt},
boxstyle/.style={rectangle, inner sep = 2pt, outer sep = 0pt, draw, fill=white}
}

\lstdefinestyle{tikzstyle}{
  % Base language
  language=[LaTeX]TeX,
  % Basic font setup
  basicstyle=\ttfamily\small,
  % TikZ keywords
  morekeywords={
    tikzpicture,draw,fill,shade,filldraw,
    node,coordinate,path,foreach,matrix,scope,clip
  },
  keywordstyle=\color{blue}\bfseries,
  % Comments & strings
  commentstyle=\color{green!50!black}\itshape,
  stringstyle=\color{red!75!black},
  % Line numbers
  numbers=left,
  numberstyle=\tiny\color{gray},
  stepnumber=1,
  numbersep=5pt,
  % Frame
  frame=single,
  framesep=4pt,
  rulecolor=\color{black},
  % Line breaking
  breaklines=true,
  breakatwhitespace=true,
  % No visible spaces/tabs
  showspaces=false,
  showstringspaces=false,
  tabsize=2,
}

% Set this style as the default for all listings:
\lstset{style=tikzstyle}
